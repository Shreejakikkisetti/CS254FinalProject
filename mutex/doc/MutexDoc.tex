\documentclass{article}
\usepackage{listings}
\usepackage[margin=1in]{geometry}

\title{TLA+ Mutex Lock Specification}
\author{CS254 Final Project}
\date{\today}

\begin{document}
\maketitle

\section{Overview}
This document explains the TLA+ specification of a mutex lock system with FIFO queuing. The specification ensures both safety (mutual exclusion) and liveness (no starvation) properties.

\section{State Space}
The specification uses three state variables:
\begin{itemize}
    \item pc: Program counter for each process (noncritical, trying, critical)
    \item lock: Lock state (0 for free, or process ID that holds it)
    \item queue: Sequence of processes waiting for the lock
\end{itemize}

\section{Actions}
\subsection{Try}
When a process wants to enter the critical section:
\begin{itemize}
    \item Changes state from noncritical to trying
    \item Appends itself to end of queue
\end{itemize}

\subsection{Enter}
A process can enter the critical section when:
\begin{itemize}
    \item It is in trying state
    \item The lock is free (lock = 0)
    \item It is at the head of the queue
\end{itemize}

\subsection{Exit}
When a process is done:
\begin{itemize}
    \item Releases the lock
    \item Returns to noncritical state
\end{itemize}

\section{Properties}
\subsection{Safety}
Mutual Exclusion: No two processes can be in the critical section simultaneously.
\begin{verbatim}
MutualExclusion ==
    \A p1, p2 \in Procs :
        (p1 # p2) => ~(pc[p1] = "critical" /\ pc[p2] = "critical")
\end{verbatim}

\subsection{Liveness}
No Starvation: If a process is in the queue, it eventually enters the critical section.
\begin{verbatim}
NoStarvation ==
    \A p \in Procs : InQueue(p) ~> (pc[p] = "critical")
\end{verbatim}

\section{FIFO Queue Implementation}
The specification uses a sequence to implement strict FIFO ordering:
\begin{itemize}
    \item New processes are added to end: Append(queue, p)
    \item Only head of queue can enter: Head(queue) = p
    \item Process removed from front when entering: Tail(queue)
\end{itemize}

This ensures processes enter the critical section in exactly the order they requested access.

\section{Fairness}
The specification includes weak fairness conditions for all actions:
\begin{verbatim}
Fairness == \A p \in Procs :
    /\ WF_vars(Try(p))
    /\ WF_vars(Enter(p))
    /\ WF_vars(Exit(p))
\end{verbatim}

This ensures that if an action is continuously enabled, it will eventually be taken.

\end{document}
